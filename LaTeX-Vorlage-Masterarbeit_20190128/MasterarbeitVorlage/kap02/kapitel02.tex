\section{Gestaltung flexibler Stromtarife f"ur den Prosumer}
Durch technologischen Fortschritt und die M"oglichkeiten f"ur Privathaushalte zur dezentralen Energieversorgung beizutragen ergeben sich neue Potenziale und Bedarfe nach innovativen tariflichen Vertragsmodellen zwischen den Akteuren. Die Rolle des Retailers ist prinzipiell "ahnlich wie bisher gestaltet: Am Spotmarket f"ur Strom der EEX-B"orse wird im Rahmen der Day-Ahead-Auktion Strommengen erworben und an einen Endkunden zu einem fixierten Preis weitergehandelt. Der Kunde nimmt nun aber eine neue Funktion ein aufgrund der F"ahigkeit selbst Energie zu erzeugen -- beispielsweise Solar- und Windenergie oder auch mit einer Kraft-W"arme-Kopplungsanlage (KWK), die durch Betrieb mit Erdgas sowohl Strom als auch W"arme als Nebenprodukt generiert. Somit hat der Prosumer \footnote{zugleich Producer und Consumer}, zus"atzlich zum herk"ommlichen Strombezug, die M"oglichkeit zur energetischen Selbstversorgung oder Strom aus seinem Microgrid, das an das Verbundnetz angeschlossen ist, zu exportieren. Das bedeutet f"ur ihn eine  Einnahmequelle, zumal seine Energieproduktion staatlich bezuschusst wird. Weiterhin ist dies auch eine Chance zur Gewinnerh"ohung auf Seite des Retailers unter alternativen Tarifbedingungen. In Perioden besonders hoher Nachfrage nach Strom, den Critical Peak Points (CPP), steigt der B"orsenpreis stark an. Durch die M"oglichkeit des Prosumers auf eigene Energiequellen zu schalten gibt es hier Einsparungspotenziale, die im Tarifvertrag genutzt werden k"onnen. So wird der Retailer f"ur die typischen kritischen Zeiten einen Aufschlag vereinbaren und ebenfalls davon profitieren. Um diesen Vertrag f"ur den Prosumer attraktiv zu machen, ist es auch sinnvoll den Basispreis geringer zu halten als im Rahmen konventioneller Stromtarife. In dieser Arbeit wollen wir nun mithilfe von Modellierung und numerischer Optimierung bestm"ogliche Tarifbedingungen f"ur diese Konstellation spezifizieren unter der Annahme, dass beide Parteien jeweils den eigenen Gewinn maximieren m"ochten. Weiterhin m"ussen einige technische Bedingungen formuliert werden, die die Einrichtung des Prosumers betreffen. In unserem vereinfachten Modell ist der Prosumer jedoch nur mit einer KWK Anlage ausgestattet sowie einer Batterie zur Speicherung von Energie und einem Boiler.