\section{Berechnung optimaler Konditionen f"ur flexible Tarifverträge}

\subsection{Rechenumgebung}

\subsection{Konvergenz und Geschwindigkeit des Verfahren}

\subsection{Diskussion der Ergebnisse und Ergebnisqualit"at} \label{results}

Die Daten f"ur den Bedarf an Strom und W"arme werden abh"angig vom Datum bezogen und der durchschnittliche B"orsenpreis errechnet. Den restlichen Parametern wurden Werte nach folgender Tabelle zugewiesen.

\begin{tabular}{l l l}
\multicolumn{3}{l}{Parameter}\\
\hline
Bezeichnung & Einheit & Wert \\
\hline
$\abat$ & & $0,0002 \Delta t$ \\
$\ahsu$ & & $0$ \\
$\gfit$ & EUR/MWh & 39,82 \\
$\ggas$ & EUR/MWh & 73,9 \\
$\gret$ & EUR/MWh & 64,2 \\
$\gsubchp$ & EUR/MWh & 54,1 \\
$\gtax$ & EUR/MWh & 234,4 \\
$\zeta$ & & 1,6 \\
$\etabatc$ & & 0,9 \\
$\etabatd$ & & 0,9 \\
$\etahsuc$ & & 1,0 \\
$\etahsud$ & & 1,0 \\
$\bar E$ & MWh & 0,0135 \\
$\underline{H}$ & MWh & 0,005973\\
$\bar H$ & MWh & 0,011092 \\
$g^{chp}$ & MWh & 0,001 \\
$k$ & & 0,4 \\
$\bpchpe$ & MWh & 0,001 \\
\end{tabular}